\documentclass[reqno]{amsart}
\usepackage{enumerate}
\usepackage{amsmath}
\usepackage{scrextend}
\usepackage{bm}
\usepackage{algorithm}
\usepackage{algpseudocode}
\usepackage{graphicx}
\usepackage{enumitem}
\usepackage{tcolorbox}
\usepackage[headheight=12pt,textwidth=7in,top=1in, bottom=1in]{geometry}
\usepackage{listings}
\usepackage{cancel}
\usepackage{color} %red, green, blue, yellow, cyan, magenta, black, white
\definecolor{mygreen}{RGB}{28,172,0} % color values Red, Green, Blue
\definecolor{mylilas}{RGB}{170,55,241}

\usepackage[]{xcolor}
\definecolor{lightblue}{rgb}{0.63, 0.74, 0.78}
\definecolor{seagreen}{rgb}{0.18, 0.42, 0.41}
\definecolor{orange}{rgb}{0.85, 0.55, 0.13}
\definecolor{silver}{rgb}{0.69, 0.67, 0.66}
\definecolor{rust}{rgb}{0.72, 0.26, 0.06}
\definecolor{purp}{RGB}{68, 14, 156}

\colorlet{lightrust}{rust!50!white}
\colorlet{lightorange}{orange!25!white}
\colorlet{lightlightblue}{lightblue}
\colorlet{lightsilver}{silver!30!white}
\colorlet{darkorange}{orange!75!black}
\colorlet{darksilver}{silver!65!black}
\colorlet{darklightblue}{lightblue!65!black}
\colorlet{darkrust}{rust!85!black}
\colorlet{darkseagreen}{seagreen!85!black}

\usepackage{hyperref}
\hypersetup{
  colorlinks=true,
}

\hypersetup{
  linkcolor=darkrust,
  citecolor=seagreen,
  urlcolor=darkrust,
  pdfauthor=author,
}

\usepackage{cleveref}

%some custom commands you may find useful
\usepackage{xparse}
\DeclareDocumentCommand{\diff}{O{} m}{
	\frac{\mathrm{d} #1}{\mathrm{d}#2}
}
\DeclareDocumentCommand{\difftwo}{O{} m}{
	\frac{\mathrm{d}^2 #1}{\mathrm{d}#2^2}
}
\DeclareDocumentCommand{\pdiff}{O{} m}{
	\frac{\partial #1}{\partial #2}
}
\DeclareDocumentCommand{\pdifftwo}{O{} m}{
	\frac{\partial^{2} #1}{\partial #2^{2}}
}
\DeclareDocumentCommand{\integral}{O{} O{} m O{x}}{
	\int_{#1}^{#2} #3\ \mathrm{d}#4
}
\DeclareDocumentCommand{\sp}{}{
	\qquad \qquad \qquad }{
}

\DeclareDocumentCommand{\bU}{}{\bm{U}}
\DeclareDocumentCommand{\bE}{}{\bm{E}}
\DeclareDocumentCommand{\bF}{}{\bm{F}}
\DeclareDocumentCommand{\bD}{}{\bm{D}}
\DeclareDocumentCommand{\bA}{}{\bm{A}}
\DeclareDocumentCommand{\bB}{}{\bm{B}}
\DeclareDocumentCommand{\bI}{}{\bm{I}}

\NewDocumentEnvironment{solution}{}{
	\begin{addmargin}[2em]{0pt}
	}{\end{addmargin} \vskip0.25cm
}

\newenvironment{sysmatrix}[1]
{\left(\begin{array}{@{}#1@{}}}
	{\end{array}\right)}
\newcommand{\ro}[1]{%
	\xrightarrow{\mathmakebox[\rowidth]{#1}}%
}
\newlength{\rowidth}% row operation width
\AtBeginDocument{\setlength{\rowidth}{3em}}

\def\name{Ben Wilfong} %your name goes here
\def\ID{bwilfong3} %your cm goes here

%these packages create the footer and page numbering
\usepackage{fancyhdr}
\usepackage{lastpage}
\pagestyle{fancy}
\lhead{\name}
%%%%%%%%%%%%%%%%%%%%%%%%%%%%%%%%
\chead{ME 7751 Homework \#1}
%%%%%%%%%%%%%%%%%%%%%%%%%%%%%%%%
\rhead{Username: \ID}
\fancyfoot[C]{\footnotesize Page \thepage\ of \pageref{LastPage}}
\fancypagestyle{firststyle}
{ \renewcommand{\headrulewidth}{0pt}%
	\fancyhf{}%
	\fancyfoot[C]{\footnotesize Page \thepage\ of \pageref{LastPage}}
}

\lstset{frame=tb,
  language=[90]Fortran,
  aboveskip=3mm,
  belowskip=3mm,
  showstringspaces=false,
  columns=flexible,
  basicstyle={\footnotesize\ttfamily},
  numbers=none,
  numberstyle=\tiny\color{gray},
  keywordstyle=\color{darklightblue},
  commentstyle=\color{seagreen},
  stringstyle=\color{darkrust},
  breaklines=true,
  breakatwhitespace=true,
  tabsize=3,
}

\begin{document}
	\noindent
	\thispagestyle{firststyle}
	\begin{tabular}{l}
		{\LARGE \textbf{ME 7751: Intro to CFD} }\\
		%%%%%%%%%%%%%%%%%%%%%%%%%
		{\Large Homework Set \#1}
		%%%%%%%%%%%%%%%%%%%%%%%%%
	\end{tabular} \hfill \begin{tabular}{r}
		\name \\
		Username: \ID
	\end{tabular}
	\noindent\makebox[\linewidth]{\rule{\textwidth}{1pt}}

    \section{Numerical Method}
    \noindent The Navier--Stokes equations with artificial compressibility are given by
    \begin{equation*}
        \pdiff[\bU]{t} + \pdiff[\bE]{x} + \pdiff[\bF]{y} =
        \frac{1}{\mathrm{Re}}\left(\pdifftwo{x} + \pdifftwo{y}\right)\left[\bD\right]\bU
    \end{equation*}
    where
    \begin{equation*}
        \bU = \begin{bmatrix} p \\ u \\ v \end{bmatrix}, \quad
        \bE = \begin{bmatrix} u/\beta \\ u^2 + p \\ uv \end{bmatrix}, \quad
        \bF = \begin{bmatrix} v/\beta \\ uv \\ v^2 + p \\ \end{bmatrix}, \quad
        \bD = \begin{bmatrix} 0 & 0 & 0 \\ 0 & 1 & 0 \\ 0 & 0 & 1 \end{bmatrix}.
    \end{equation*}
    This approach to solving the incompressible Navier--Stokes equations workd because at steady state, $\partial p/\partial t = 0$ and the continuity eqation ensures the velocity field is divergence free.
    Rearranging the terms gives
    \begin{equation*}
        \pdiff[\bU]{t} = \bm{F}(\bU) \quad \text{where} \quad \bm{F}(\bU) =
        \frac{1}{\mathrm{Re}}\left(\pdifftwo{x} + \pdifftwo{y}\right)\left[\bD\right]\bU
        - \pdiff[\bE]{x} - \pdiff[\bF]{y}.
    \end{equation*}
    The equations are integrated in time using the trapezoidal rule
    \begin{align*}
        \bU^{n+1} &= \bU^n +
            \frac{\Delta t}{2}\left[
                \bm{F}(\bU^n) + \bm{F}(\bU^{n+1})
            \right], \\
                  &= \bU^n +
            \frac{\Delta t}{2}\left[
                \left(\frac{1}{\mathrm{Re}}\left(\pdifftwo{x} + \pdifftwo{y}\right)\left[\bD\right]\bU - \pdiff[\bE]{x} - \pdiff[bF]{y}\right)^n +
                \left(\frac{1}{\mathrm{Re}}\left(\pdifftwo{x} + \pdifftwo{y}\right)\left[\bD\right]\bU - \pdiff[\bE]{x} - \pdiff[\bF]{y}\right)^{n+1}
            \right].
    \end{align*}
    Taylor expanding $\bE$ and $\bF$ about the time level $n$ gives
    \begin{align*}
        \bE^{n+1} = \bE^n + [\bm{A}]^n\left(\bU^{n+1} - \bU^n\right), \\
        \bF^{n+1} = \bF^n + [\bm{B}]^n\left(\bU^{n+1} - \bU^n\right),
    \end{align*}
    where
    \begin{equation*}
        [\bA]^n = \begin{bmatrix} 0 & 1/\beta & 0 \\ 1 & 2u & 0 \\ 0 & v & u \end{bmatrix}, \quad
        [\bB]^n = \begin{bmatrix} 0 & 0 & 1/\beta \\ 0 & v & u \\ 1 & 0 & 2v \end{bmatrix}.
    \end{equation*}
    Substituting the Taylor expansions of $\bE$ and $\bF$ gives
    \begin{equation*}
        \begin{aligned}
            \bU^{n+1} = \bU^n + &\frac{\Delta t}{2}
        \left[
        \frac{1}{\mathrm{Re}}\left(\pdifftwo{x}  + \pdifftwo{y}\right)[\bD]\bU^n
        -\pdiff[\bE^n]{x} - \pdiff[\bF^n]{Y}
        + \frac{1}{\mathrm{Re}}\left(\pdifftwo{x} + \pdifftwo{y}\right)[\bD]\bU^{n+1}
        -\pdiff[\bE^n]{x} - \pdiff[\bF^n]{y} \dots
        \right. \\
        &\left.
        - \pdiff[[\bA]^n\bU^{n+1}]{x} - \pdiff[[\bB]^n\bU^{n+1}]{y}
        + \pdiff[[\bA]^n\bU^n]{x} + \pdiff[[\bB]^n\bU^n]{y}
        \right]
        \end{aligned}
    \end{equation*}
    Rearranging and simpliying gives
    \begin{equation*}
        \begin{aligned}
            \left\{[\bI] + \frac{\Delta t}{2}\pdiff[[\bA]^n]{x}\right. &+ \left.\frac{\Delta t}{2}\pdiff[[\bB]^n]{y} + \frac{\Delta t}{2\mathrm{Re}}\left(\pdifftwo{x} + \pdifftwo{y}\right)[\bD]\right\}\bU^{n+1} = \\
           &\underbrace{\left\{[\bI] - \frac{\Delta t}{2}\pdiff[[\bA]^n]{x} - \frac{\Delta t}{2}\pdiff[[\bB]^n]{y} + \frac{\Delta t}{2\mathrm{Re}}\left(\pdifftwo{x} + \pdifftwo{y}\right)[\bD]\right\}\bU^n + \Delta t\left(\pdiff[\bE^n]{x} + \pdiff[\bF^n]{y}\right)}_{\text{RHS}}.
        \end{aligned}
    \end{equation*}
    An approximate factorization of the left hand side gives the equation
    \begin{equation*}
        \left([\bI] + \frac{\Delta t}{2}\pdiff[[\bA]^n]{x} + \frac{\Delta t}{2\mathrm{Re}}\pdifftwo[[\bD]]{x}\right)
        \left([\bI] + \frac{\Delta t}{2}\pdiff[[\bB]^n]{y} + \frac{\Delta t}{2\mathrm{Re}}\pdifftwo[[\bD]]{y}\right)\bU^{n+1} = \text{RHS}.
    \end{equation*}
    This approximate factorization results in additional terms that are not present in the original equations.
    These terms are multiplied by $\Delta t^2$ and can be neglected.
    The above can be written in delta form, where $\Delta \bU = \bU^{n+1} - \bU^N$, as
    \begin{equation*}
        \left([\bI] + \frac{\Delta t}{2}\pdiff[[\bA]^n]{x} + \frac{\Delta t}{2\mathrm{Re}}\pdifftwo[[\bD]]{x}\right)
        \left([\bI] + \frac{\Delta t}{2}\pdiff[[\bB]^n]{y} + \frac{\Delta t}{2\mathrm{Re}}\pdifftwo[[\bD]]{y}\right)\Delta \bU = \Delta t\left(\pdiff[\bE^n]{x} + \pdiff[\bF^n]{y}\right).
    \end{equation*}
    The equations can then be advanced in time using the following steps:
    \begin{enumerate}
        \item Solve for $\Delta \bU'$ by solving
            \[\left([\bI] + \frac{\Delta t}{2}\pdiff[[\bA]^n]{x} + \frac{\Delta t}{2\mathrm{Re}}\pdifftwo[[\bD]]{x}\right)\Delta \bU' = \Delta t\left(\pdiff[\bE^n]{x} + \pdiff[\bF^n]{y}\right).\]
        \item Solver for $\Delta \bU$ by solving
            \[\left([\bI] + \frac{\Delta t}{2}\pdiff[[\bB]^n]{y} + \frac{\Delta t}{2\mathrm{Re}}\pdifftwo[[\bD]]{y}\right)\bU = \bU'.\]
        \item Update the solution by performing $\bU^{n+1} = \bU^n + \Delta \bU$.
    \end{enumerate}
    Writing the implicit time-stepping in this manner is advantageous because it results in tridiagonal systems that can be solved in $\mathcal{O}(n)$ operations using a tri-diagonal matrix algorithm (TDMA).
    The model is closed using the artifial equation of state $p = \rho/\beta$.
    Steady stat is defined to be reached when
    \begin{equation*}
        \sum_{i,j}\sqrt{\frac{1}{N^2}\left(\pdiff[\rho_{i,j}]{t}\right)} \le \epsilon
    \end{equation*}
    for some chosen tolerance $\epsilon$.
    All derivatives are calculated using the second-order central differences
    \begin{equation*}
        \pdiff[f]{x} = \frac{f_{i+1} - f_{i-1}}{2\Delta x}, \quad
        \pdifftwo[f]{x} = \frac{f_{i+1} - 2f_i + f_{i-1}}{\Delta x^2}.
    \end{equation*}
    The sides and bottom of the domain are set to no-slip boundary conditions where $u = v = 0$ and $\rho = 1$.
    The top of the domain is set to a moving boundary condition where $u = 1, v = 0,$ and $\rho = 1$.
    The finite difference points at the boundaries are set to the values of the boundary conditions.
    No ghost cell regions are required because the stencils for the interior cells do not require values outside of the boundary cells.
    The height and width of the domain are fixed to $L = 1$ and the desired Reynolds number is obtained by varying $\nu$.
\end{document}

